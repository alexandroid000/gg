\documentclass[]{article}
\usepackage{lmodern}
\usepackage{amssymb,amsmath}
\usepackage{ifxetex,ifluatex}
\usepackage{fixltx2e} % provides \textsubscript
\ifnum 0\ifxetex 1\fi\ifluatex 1\fi=0 % if pdftex
  \usepackage[T1]{fontenc}
  \usepackage[utf8]{inputenc}
\else % if luatex or xelatex
  \ifxetex
    \usepackage{mathspec}
  \else
    \usepackage{fontspec}
  \fi
  \defaultfontfeatures{Ligatures=TeX,Scale=MatchLowercase}
\fi
% use upquote if available, for straight quotes in verbatim environments
\IfFileExists{upquote.sty}{\usepackage{upquote}}{}
% use microtype if available
\IfFileExists{microtype.sty}{%
\usepackage[]{microtype}
\UseMicrotypeSet[protrusion]{basicmath} % disable protrusion for tt fonts
}{}
\PassOptionsToPackage{hyphens}{url} % url is loaded by hyperref
\usepackage[unicode=true]{hyperref}
\hypersetup{
            pdftitle={CS 476 Intermediate Report},
            pdfauthor={Alli Nilles (nilles2) and Spencer Gordon (slgordo2)},
            pdfborder={0 0 0},
            breaklinks=true}
\urlstyle{same}  % don't use monospace font for urls
\usepackage[margin=2cm]{geometry}
\IfFileExists{parskip.sty}{%
\usepackage{parskip}
}{% else
\setlength{\parindent}{0pt}
\setlength{\parskip}{6pt plus 2pt minus 1pt}
}
\setlength{\emergencystretch}{3em}  % prevent overfull lines
\providecommand{\tightlist}{%
  \setlength{\itemsep}{0pt}\setlength{\parskip}{0pt}}
\setcounter{secnumdepth}{0}
% Redefines (sub)paragraphs to behave more like sections
\ifx\paragraph\undefined\else
\let\oldparagraph\paragraph
\renewcommand{\paragraph}[1]{\oldparagraph{#1}\mbox{}}
\fi
\ifx\subparagraph\undefined\else
\let\oldsubparagraph\subparagraph
\renewcommand{\subparagraph}[1]{\oldsubparagraph{#1}\mbox{}}
\fi

% set default figure placement to htbp
\makeatletter
\def\fps@figure{htbp}
\makeatother

\usepackage{jeffe, graphicx, mathtools, tikz, xspace}
\usetikzlibrary{arrows.meta}
\usepackage[charter]{mathdesign}
\makeatother
\DeclareMathOperator{\attach}{attach}
\DeclareMathOperator{\relabel}{relabel}
\DeclareMathOperator{\detach}{detach}
\def\RuleSeq{\mathrm{RuleSeq}\xspace}
\def\Graphs{\mathcal{G}\xspace}
\def\Concepts{\mathcal{C}\xspace}
\def\Hypotheses{\mathcal{H}\xspace}
\def\ruleseq{\rho\xspace}

\title{CS 476 Final Report}
\author{Alli Nilles (\texttt{nilles2}) and Spencer Gordon (\texttt{slgordo2})}
\date{}

\begin{document}
\maketitle

\newcommand{\step}[1]{\xrightarrow{#1}}
\newcommand{\steps}[1]{\xRightarrow{#1}}

% Brainstorm May 4
% ==================

% Related Work
% ============

% ### Learning compact rule sets for generating a given graph



% ### Decidability of termination

% Old
% ===


In our project proposal, motivated by applications in self-assembling
robotic system, we asked ``Might it be possible to learn compact rule
sets that generate stable assemblies, or dynamics on graphs that are
recurrent?''

We then proposed a framework for learning functions from rule sets to
steady-state graph dynamics. In the course of the project, we have so
far focused on surveying known computational learning theory results for
learning grammars, as well as surveying known results in the theory of
graph grammars.

We show that a particular formulation of a concept class is inherently
unpredictable, under the definition from \cite{kearns1994}:

\begin{quote}
A concept class \(C\) is \emph{inherently unpredictable} if the VC
dimension of \(C_n\) is polynomial in \(n\), yet \(C\) is not
efficiently PAC learnable using any polynomially evaluatable hypothesis
class \(H\).
\end{quote}

Specifically, we show that learning graph grammars which lead to to
stable, connected assemblies is inherently unpredictable by reduction to
learning DFAs. We also outline the next areas of investigation in this
project.

\section{Refinement of Project
Scope}\label{refinement-of-project-scope}

We first recall the definition of graph grammars \cite{litovsky}
\cite{klavins}:

We consider \emph{simple labelled graphs}, \(G = (V,E,l)\) where \(V\)
is an indexed set of vertices, \(E\) a set of edges between pairs from
\(V\), and \(l: V \to \Sigma\) is a labelling or coloring function mapping to a
finite alphabet $\Sigma$.

A \textbf{rule} is a pair of graphs \(r=(L,R)\) (for \emph{left} and
\emph{right}), where \(V_L = V_R\). A \emph{unary} rule has a vertex set
of size one, a \emph{binary} rule is on two vertices, etc. A rule can
change labels and add/delete edges, but cannot add or remove nodes.
Rules also require (implicit or explicit) \emph{embedding mechanisms}
which specify how these graphs should affect connections to a larger
graph in which they are embedded.

A \textbf{grammar} is a set of rules. These rules are applied
 to a graph to define a transition system. Transitions occur
when the left hand side of a rule is isomorphic to a subgraph, replacing it 
with the right hand side of the rule.

\subsubsection{Context-Free vs.~Context-Sensitive Graph
Grammars}\label{context-free-vs.context-sensitive-graph-grammars}

If we relax the assumption that rules cannot add or remove vertices, we
can form a natural division of graph grammars into context-free and
context-sensitive grammars. Context-free grammars take a single labelled
node as the left hand side of rules, and rewrite it to an arbitrary graph on
the right hand side of the rule. These grammars are also
often restricted to be \emph{confluent}, such that reordering a
sequence of rule applications does not change the outcome of the graph
transformation. This area has been the focus of the majority of
attention in the graph grammars community thus far.

Motivated by self-assembling robotic systems, we focus on context-sensitive
graph grammars: often, we have a collection of $n$ robots and wish to synthesize a
set of locally executable rules, which when applied to an entire collection of
robots, form a connected assembly with certain topological or connectivity
properties. We do not want to have to create or destroy robots in the course of
this task, and thus require context-sensitive graph grammars.

\subsection{Our concept class of
  interest}\label{our-concept-class-of-interest}

We have so far focused on a very restricted model of a context-sensitive graph
grammar, which we describe completely here.

In order to start talking about \emph{learning} graph grammars, we need a way of
taking a graph grammar and using it to classify an input graph. We came up with
the idea of classifying a problem instance - given by an input graph and a
grammar - according to the graph's connectivity after some reasonable number of transformations.

We consider labeled graphs $G = (V,E,\ell)$ as above where $V = [k]$ for some
$k\in \N$. Let $\Graphs_{k,\Sigma}$ be the set of all such labeled graphs with
$V=[k]$ and labels over the alphabet $\Sigma$.

If we allowed only a fixed number of transformations according to a grammar for
any graph, graphs above a certain size would be indistinguishable for any given
concept, since there simply wouldn't be enough time for a rule to apply to each
vertex in the graph and form one connected component.

As a result, we are considering a concept class in which the number of allowed
operations was a function of the size of the graph.

\subsubsection{Concept Class Definition}

A \textbf{rule} $r \in R_{\Sigma}$ is a pair of graphs
$r \in \Graphs_{k,\Sigma}^2$ with $k \leq 3$.

Two graphs $G,G' \in \Graphs$ are related by the \textbf{application} of rule
$r = (H,H')$, written $G \step{r} G'$ if there exists a subgraph $J \subset G$
isomorphic to $H$ and $G'$ is equal to $G$ with $J$ replaced by $J'$ where $J'$
is isomorphic to $H'$ under the same permutation for which $J$ was isomorphic to
$H$. Moreover, it has to be the case that if there all multiple such sugraphs
$J$ isomorphic to $H$, the lexicographically least such one is chosen.
In other words, $G \step{r} G'$ if you can get $G'$ by replacing the
lexicographically least $H$-isomorphic subgraph of $G$ with a corresponding
$H'$-isomorphic graph.

The \textbf{set of all sequences of rules} is given by
$\RuleSeq_{\Sigma} = R_{\Sigma}^*$. We define the semantics of a rule sequence
$\rho = r_1r_2\dotsm r_m \in \RuleSeq_{\Sigma}$ as follows: We define a relation
$\step{\rho}$ such that $G \step{\rho} G'$ if and only if there exists an
$i\leq m$ such that $G\step{r_i} G'$ and there does not exist a subgraph
$H \subseteq G$ isomorphic to the left-hand side of any rule $r_j$ for all
$j < i$. Intuitively, two graphs are related by $\rho$ if they differ by the
application of the lexicographically least rule in $\rho$ that is applicable to
$G$. In this way, we define an \emph{ordering} of the rules so that we always apply
the first rule that matches the current graph.

To write down a rule over graphs of size $3$ with labels from $\Sigma$, we will
introduce notation to describe small graphs compactly. For a graph on three
vertices we can write down a sequence of three labels $ABC \in \Sigma$ and add
the $|$ character between pairs of symbols or after the last symbol to indicate
that there is no edge between the pair of vertices around the $|$ or between the
first and last vertex, respectively.

Thus, $AAA$ denotes a triangle graph with every vertex labeled $A$, and $A|A|A|$
denotes the graph of 3 vertices labeled $A$ with no edges. Under this notation,
the following graphs are isomorophic: $A|AA = AA|A = AAA|$.

We can define an iterated version of the relation $\step{\rho}$, denoted
$\steps{\rho}_i$, where $G \steps{\rho}_i G'$ if there is a sequence of exactly
$i$ steps $G$ transforming $G$ into $G'$ under $\rho$.

Finally, since there is a unique next graph under this model of graph grammar
(since the rule application is deterministic) we can treat $\rho$ as a function,
and write $\rho(G)$ for the graph obtained by a single step under $\rho$. We'll
define $\rho(G) =G$ if no rules in $\rho$ apply to $G$. We'll write $\rho^i(G)$
for the $i$-th iterated application of $\rho$ to $G$.

We now define a function $f_{\rho,c} : \Graphs_{\Sigma} \to \Set{0,1}$ as follows:
\[f_{\rho,c}(G) = \begin{cases}
    1 &\,\text{if $\rho^{c\Card{V(G)}}(G) = \rho^{c\Card{V(G)}+1}(G)$ and $\rho^{c\Card{V(G)}}(G)$ is connected}\\
    0 &\,\text{otherwise}
  \end{cases}\]

Our concept class will be
\(\Concepts_{\Sigma,c} = \Setbar{f_{\rho,c}}{\rho\in \RuleSeq_\Sigma}\).

As in the proof of Kearns and Vazirani that DFA are inherently unpredictable, we
consider the restriction of our concept class to inputs of a given size $n$,
where $n$ is the number of vertices of the graph classified by a concept. We
write this restriction as $\Concepts_{\Sigma,c,n}$.

With this definition, the first question we wanted to address was whether this
concept class was PAC-learnable. We will now show that it is at least as
hard as learning the concept class of DFAs since we can reduce learning a DFA to
learning a graph grammar of this form.

\section{Reduction from learning DFAs}

\textbf{Definition \cite{kearns1994}:} a concept class \(C\) over
instance space \(X\) PAC-reduces to the concept class \(C'\) over
instance space \(X'\) if the following conditions are met:

\begin{itemize}
\tightlist
\item
  \emph{Efficient Instance Transformation:} There exists a mapping
  \(G: X_n \to X'_{p(n)}\) for all \(n\) and some polynomial \(p\) which
  is computable in polynomial time.
\item
  \emph{Existence of Image Concept:} For all \(c \in C_n\), there is a
  concept \(c' \in C'_{p(n)}\) such that \(size(c') \leq q(size(c))\)
  for some polynomials \(p\) and \(q\). Additionally, for all
  \(x \in X_n\), \(c(x) = 1\) if and only if \(c'(G(x)) = 1\).
\end{itemize}

Since the concept class $\Concepts_{\Sigma,c,n}$ is isomorphic to $\Concepts_{\Delta,c,n}$ for any $\Card{\Sigma} = \Card{\Delta}$, we'll write $\Concepts_{l,c,n}$ to denote the concept class for any label alphabet of size $l$. 

\textbf{Theorem:} the class of deterministic finite automata with $m$ states
over an alphabet of size $l$ PAC-reduces to the class $\Concepts_{l+4+m,2,n}$ (for any input size $n$),
i.e., the class of concepts corresponding to the connectivity of graph grammars
after $2\Card{V(G)}$ steps, over alphabets of size $l+m+4$.

\textbf{Proof:}
We describe for each DFA $D=(Q,\Sigma,\delta,q_0,F)$ a polynomially sized ruleset that
simulates \(D\) on transformed instances.

We'll transform strings in $\Sigma^n$ into graphs with $2n+2$ vertices over the alphabet $Q\cup\Sigma\cup \Set{A,T,Z}$ where $A,T,Z$ are symbols distinct from any in $\Sigma\cup Q$.

\textbf{Instance Transformation:} 
Given a string $x = x_1x_2\dotsm x_n$, we let $G = ([2n+2],E,\ell)$ be the graph where

$\ell(2i+1) = x_i$, $\ell(2i) = A$ for $i < n+1$, $\ell(1) = q_0$ and $\ell(2n+2) = T$.

We choose the edges as follows: $E = \Set{(2i,2i+2),(2i,2i+1)}{i\in [n]} \cup \Set{(1,2)}$.

\textbf{Image Concept:}
We now define the rule sequence $\rho$ as follows:

For every state $q\in Q$ and every symbol $a\in \Sigma$ in the DFA, we add a rule corresponding to the transition $\delta(q,a)$.

Particularly, we add a rule of the form $(qAa|, A|\delta(q,a)|A|)$, simulating
each of the non-terminating transitions of the DFA. The order of
these rules doesn't matter since only one of them can be active at a time.

We then add a rule $(qT, AZ)$ for each $q\in F$, and a rule $(A|Z,AZ)$, thus
simulating each of the transitions to an accepting state in the DFA, and
guaranteeing that an accepting state corresponds to a connected graph.

The resulting concept and instance satisfy the requirements for the reduction,
proving the result above.

\subsection{Questions}
In order to complete the proof that our concept class is inherently unpredictable (and thus unlikely to be PAC-learnable assuming standard cryptographic hardness assumptions), we would need to show that our concept class has VC-dimension that grows polynomially with the input size, and with size of the concept. We attempted to do this but ran into many problems that proved insurmountable. Most of these problems are the result of the strangeness of our concept class, which does not seem particularly natural and is hard to analyze and understand. An easier task that was achievable for DFA but which we couldn't replicate for our concept class was that of identifying a polynomially-bounded witness that is a counterexample to the claim that a hypothesized concept is equivalent to a target concept. With DFA, there is an algorithm running in polynomial time for finding a string for which two DFA disagree assuming they are not equivalent. We were not able to find such an algorithm for our concept class.

Having failed in our attempts to establish the properties enjoyed by DFAs for our concept class, we tried to attack the other direction of the problem, by showing that under certain assumptions this class was PAC-learnable. Since DFAs can be learned with membership and equivalence queries, we attempted to carry through a similar result for our concept class. Unfortunately, the DFA result requires efficient counterexamples to be generated by equivalence queries, which we couldn't do for our concept class.

We next tried to extend the model of learning to allow for access to an oracle that will take an input graph and rewrite it by applying one step of the rewriting relation. While under this assumption, it's clear that we can learn the set of rules present in a given concept in time polynomial in the maximum arity of the rules, we couldn't do this without being exponential in the number of labels on vertices. Moreover, we couldn't figure out how to determine the ordering or multiplicity of rules once we knew the underlying set of rules present, since the same rule could be repeated multiple times to take effect at different stages of the rewriting procedure.

Given our failures to come up with any sort of characterization of our concept class that would allow for proving the sorts of results that would be needed to come up with an algorithm for learning the class, we next considered limiting the concept class even further by considering only binary rewrite rules. We believe that this is still hard, and that the reduction from DFA learning to this concept class would still go through with only minor modifications, but we weren't able to complete the reduction in the time remaining after deciding to think about this question. 

On the other hand, it seems likely that we should be able to learn this concept class if we allow single-step queries, along with membership and equivalence queries, though again we weren't able to prove this.

Another more fundamental question is whether or not all ``minimal'' rule sequences are distinguishable. Clearly, a rule sequence that contains the same rule repeated consecutively is equivalent to a rule sequence in which no rule is repeated consecutively, but given any two such rule sequences, are there guaranteed to be graphs on which they act differently. For DFA, any two minimal DFA disagree on some string of reasonable length. We don't believe that this is the case for our concept class, but again haven't been able to prove it. The intuition for why this shouldn't be the case is that the following ruleset wold seem to be a counterexample: Pick some arbitrary ordering of the labels in the given alphabet. Now the rule sequence should consists of all ternary rules that take triples of vertices and fully connect them while updating the label of each involved vertex to some later label in the ordering. Such a rule sequence should result in the graph being fully connected, no matter the order of the rules involved, and would witness the fact that there are distinct minimal rule sequences that behave identically on all graphs.

If that is indeed the case, we can then ask whether it's possible to come up with equivalence classes of rule sequences, and learn cannonical representatives from each equivalence class. We do not know the answer to this question. 

Given the difficulties presented by our concept class, it's worth considering whether an alternative concept class would be better suited to exploring the learnability of graph rewriting rules. To address, this we recall the desiderata any reasonable hypothesis class would have. First, it should be possible to evaluate a given hypothesis on an instance in polynomial time. This turned out to be the biggest issue for our project, since finding such a concept/hypothesis class was already difficult for graph rewriting systems which weouldn't even be guaranteed to terminate upon an instance if allowed to continue operating indefinitely. Second, the class should be relatively robust, and not be particularly sensitive to incidental details of our model. This we were unable to achieve, and our concept class is very fragile, and two rule sequences that differ by the addition of one useless step can differ in the number of steps they take to converge, so that one could classify an instance positively and the other negatively, by just having one take slightly more than $c\Card{V(G)}$ steps, and having the other take slightly less than $c\Card{V(G)}$ steps. As a result, we have little confidence in the generalizability of any of our results. We weren't to come up with a more robust concept class that better fit into the PAC-learning framework, even with the addition of equivalence and membership queries (or other additional queries we considered to make the problem easier).

\subsection{Next Steps}

In order to show that these concepts are inherently unpredictable, we next have
to show that they have polynomial VC-dimension. This is the next question we'd
like to address. After doing this, we would want to see if we can get a positive
result showing the PAC-learnability of this concept class under a stronger
learning model, perhaps in which the learner can make membership queries or
equivalence queries.


\bibliographystyle{ieeetr}
\bibliography{refs}

\end{document}
