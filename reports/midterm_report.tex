\documentclass[]{article}
\usepackage{lmodern}
\usepackage{amssymb,amsmath}
\usepackage{ifxetex,ifluatex}
\usepackage{fixltx2e} % provides \textsubscript
\ifnum 0\ifxetex 1\fi\ifluatex 1\fi=0 % if pdftex
  \usepackage[T1]{fontenc}
  \usepackage[utf8]{inputenc}
\else % if luatex or xelatex
  \ifxetex
    \usepackage{mathspec}
  \else
    \usepackage{fontspec}
  \fi
  \defaultfontfeatures{Ligatures=TeX,Scale=MatchLowercase}
\fi
% use upquote if available, for straight quotes in verbatim environments
\IfFileExists{upquote.sty}{\usepackage{upquote}}{}
% use microtype if available
\IfFileExists{microtype.sty}{%
\usepackage[]{microtype}
\UseMicrotypeSet[protrusion]{basicmath} % disable protrusion for tt fonts
}{}
\PassOptionsToPackage{hyphens}{url} % url is loaded by hyperref
\usepackage[unicode=true]{hyperref}
\hypersetup{
            pdftitle={CS 476 Intermediate Report},
            pdfauthor={Alli Nilles (nilles2) and Spencer Gordon (slgordo2)},
            pdfborder={0 0 0},
            breaklinks=true}
\urlstyle{same}  % don't use monospace font for urls
\usepackage[margin=2cm]{geometry}
\IfFileExists{parskip.sty}{%
\usepackage{parskip}
}{% else
\setlength{\parindent}{0pt}
\setlength{\parskip}{6pt plus 2pt minus 1pt}
}
\setlength{\emergencystretch}{3em}  % prevent overfull lines
\providecommand{\tightlist}{%
  \setlength{\itemsep}{0pt}\setlength{\parskip}{0pt}}
\setcounter{secnumdepth}{0}
% Redefines (sub)paragraphs to behave more like sections
\ifx\paragraph\undefined\else
\let\oldparagraph\paragraph
\renewcommand{\paragraph}[1]{\oldparagraph{#1}\mbox{}}
\fi
\ifx\subparagraph\undefined\else
\let\oldsubparagraph\subparagraph
\renewcommand{\subparagraph}[1]{\oldsubparagraph{#1}\mbox{}}
\fi

% set default figure placement to htbp
\makeatletter
\def\fps@figure{htbp}
\makeatother

\usepackage{jeffe, graphicx, mathtools, tikz, xspace}
\usetikzlibrary{arrows.meta}
\usepackage[charter]{mathdesign}
\makeatother
\DeclareMathOperator{\attach}{attach}
\DeclareMathOperator{\relabel}{relabel}
\DeclareMathOperator{\detach}{detach}
\def\Card\#1\{\left\textbar{}
\def\RuleSeq{\mathrm{RuleSeq}\xspace}
\def\Graphs{\mathcal{G}\xspace}
\def\Concepts{\mathcal{C}\xspace}
\def\Hypotheses{\mathcal{H}\xspace}

\title{CS 476 Intermediate Report}
\author{Alli Nilles (\texttt{nilles2}) and Spencer Gordon (\texttt{slgordo2})}
\date{}

\begin{document}
\maketitle

\newcommand{\step}[1]{\xrightarrow{#1}}

In our project proposal, motivated by applications in self-assembling
robotic system, we asked ``Might it be possible to learn compact rule
sets that generate stable assemblies, or dynamics on graphs that are
recurrent?''

We then proposed a framework for learning functions from rule sets to
steady-state graph dynamics. In the course of the project, we have so
far focused on surveying known computational learning theory results for
learning grammars, as well as surveying known results in the theory of
graph grammars.

We show that a particular formulation of a concept class is inherently
unpredictable, under the definition from {[}@kearns1994{]}:

\begin{quote}
A concept class \(C\) is \emph{inherently unpredictable} if the VC
dimension of \(C_n\) is polynomial in \(n\), yet \(C\) is not
efficiently PAC learnable using any polynomially evaluatable hypothesis
class \(H\).
\end{quote}

Specifically, we show that learning graph grammars which lead to to
stable, connected assemblies is inherently unpredictable by reduction to
learning DFAs. We also outline the next areas of investigation in this
project.

\subsection{Refinement of Project
Scope}\label{refinement-of-project-scope}

We retain the same definition of graph grammars {[}@litovsky{]}
{[}@klavins{]}:

We consider \emph{simple labelled graphs}, \(G = (V,E,l)\) where \(V\)
is an indexed set of vertices, \(E\) a set of edges between pairs from
\(V\), and \(l: V \to E\) is a labelling or coloring function.

A \textbf{rule} is a pair of graphs \(r=(L,R)\) (for \emph{left} and
\emph{right}), where \(V_L = V_R\). A \emph{unary} rule has a vertex set
of size one, a \emph{binary} rule is on two vertices, etc. A rule can
change labels and add/delete edges, but cannot add or remove nodes.
Rules also require (implicit or explicit) \emph{embedding mechanisms}
which specify how these graphs should affect connections to a larger
graph in which they are embedded.

A \textbf{grammar} is a set of rules. These rules are applied
nondeterministically to a graph to define a transition system.
Transitions occur when the left hand sides of a rule matches a subgraph
and replaces it with the right hand side of the rule.

\subsubsection{Context-Free vs.~Context-Sensitive Graph
Grammars}\label{context-free-vs.context-sensitive-graph-grammars}

If we relax the assumption that rules cannot add or remove vertices, we
can form a natural division of graph grammars is into context-free and
context-sensitive grammars. Context-free grammars take a single labelled
node and replace it with an arbitrary graph. These grammars are also
often restricted to be \emph{confluent}, such that the result of a
sequence of rule applications does not change the outcome of the graph
transformation. This area has been the focus of the majority of
attention in the graph grammars community, but context-sensitive graph
grammars have recieved relatively little.

The application to self-assembling robotic systems motivates the choice
to focus on context-sensitive graph grammars: often, we have a
collection of \(n\) robots and wish to produce a set of locally
executable rules that result in the entire collection of robots forming
a connected assembly with certain topological or connectivity
properties.

\subsubsection{Our concept class of
interest}\label{our-concept-class-of-interest}

In particular, we are interested in \emph{programmed context-sensitive
graph replacement systems}, as described in {[}@gg\_handbook{]}. This
area of research introduces control-flow statements for specifying the
order of application of rules: for instance, we may specify that while
rule 1 applies, apply it repeatedly; then apply rule 2 until rule 1 is
applicable again. Such ``programmable'' graph rewriting specifications
can be used for decision procedures.

In this paper, we will examine the case where we apply a given sequence
of rules for a number of iterations that is proportional to the size of
the overall graph. After this number of rewrites, we examine the graph -
if it is connected, we consider this an ``accepting'' state, and
otherwise we reject.

Additionally, we will consider only \emph{node-preserving} rewrites, in
which the number of nodes on the left and right sides of the rewrite are
the same. Edges may be added and removed in a rewrite, and labels may
change.

Is this concept class efficiently PAC learnable? If not, what
constraints can we add to the concept class to make it learnable?
Likewise, if not, what augmentations to the basic PAC learning model
(such as membership and equivalence queries) do we need to add to make
the concept learnable? In this fashion we are exploring the ``frontier''
of learnable concept classes and learning models in this relatively
unexplored formulation of graph grammars.

\subsection{Definitions}\label{definitions}

Let \(\RuleSeq_\Sigma\) be a sequence of rules over a finite set of
labels \(\Sigma\), where a rule is a node-preserving graph rewrite.

Let \(\Graphs_{n,\Sigma}\) denote the set of all pairs \((G,\ell)\)
where \(G\) is a graph on vertex set \([n]\) and
\(\ell : [n] \to \Sigma\) assigns a label to each vertex. Let
\(\Graphs_{\Sigma} = \cup_{n\geq 0} \Graphs_{n,\Sigma}\).

For each rule sequence \(S\in \RuleSeq\), there is a corresponding step
relation \(\step{S}\) such that \[G {\step{S}} G'\] if and only \(G\)
differs from \(G'\) by a single application of the first rule in \(S\)
that can be applied, to the lexicographically first vertex or vertices
to which it can be applied.

For any \(S\in \RuleSeq\), we define the function
\(f_S : \Graphs_{\Sigma} \to \Set{0,1}\) as follows:
\[f_S(G) = \begin{cases}
    1 &\,\text{if after $N |V(G)|$ steps of ${\step{S}}$ starting with $G$ a
    stable, connected graph is formed.}\\
    0 &\,\text{otherwise}
  \end{cases}\]

Our concept class will be
\(\Concepts_\Sigma = \Setbar{f_S}{S\in \RuleSeq_\Sigma}\). Our
hypothesis class will be \(\Hypotheses_{\Sigma} = \Concepts_\Sigma\).

\textbf{Definition {[}@kearns1994{]}:} a concept class \(C\) over
instance space \(X\) PAC-reduces to the concept class \(C'\) over
instance space \(X'\) if the following conditions are met:

\begin{itemize}
\tightlist
\item
  \emph{Efficient Instance Transformation:} There exists a mapping
  \(G: X_n \to X'_{p(n)}\) for all \(n\) and some polynomial \(p\) which
  is computable in polynomial time.
\item
  \emph{Existence of Image Concept:} For all \(c \in C_n\), there is a
  concept \(c' \in C'_{p(n)}\) such that \(size(c') \leq q(size(c))\)
  for some polynomials \(p\) and \(q\). Additionally, for all
  \(x \in X_n\), \(c(x) = 1\) if and only if \(c'(G(x)) = 1\).
\end{itemize}

\subsection{Preliminary Work}\label{preliminary-work}

\textbf{Theorem:} the class of deterministic finite automata PAC-reduces
to the class of node-conserving graph rewriting rulesets which achieve
full connectivity after \(k |V(G)|\) steps on graph \(G\).

\textbf{Proof:}

We describe for each DFA \(D\) a polynomially sized ruleset that
simulates \(D\) on transformed instances.

\subsubsection{Instance Transformation}\label{instance-transformation}

We define a transformation \(G: \{0,1\}^n \to G_{p(n)}\) from the space
of binary DFA inputs of length \(n\) to the space of graphs with a
number of vertices polynomial in \(n\).

Take an instance \(x \in \{0,1\}^n\). For each character \(s \in x\),
create a pair of vertices. Label one vertex \(a\), and label the other
with \(b_0\) if \(s=0\) and \(b_1\) if \(s=1\). Add one more vertex to
the graph and label it \(\lambda\).

An example input transformation for the input \(1011\) would be:

\begin{tikzpicture}
[vertex/.style={circle, draw, fill=black, inner sep=0pt, minimum width=4pt}]

\node[vertex] (1) [label=left:$\lambda$] {};
\node[vertex] (2) [label=left:$a$] at (0,-1) {};
\node[vertex] (3) [label=left:$a$] at (0,-2) {};
\node[vertex] (4) [label=left:$a$] at (0,-3) {};
\node[vertex] (5) [label=left:$a$] at (0,-4) {};

\node[vertex] (6) [label=right:$b_1$] at (1,-1) {};
\node[vertex] (7) [label=right:$b_0$] at (1,-2) {};
\node[vertex] (8) [label=right:$b_1$] at (1,-3) {};
\node[vertex] (9) [label=right:$b_1$] at (1,-4) {};

\end{tikzpicture}

\subsubsection{Image Concept}\label{image-concept}

We create a rule sequence which exactly simulates the allowed
transitions in the DFA.

First, assign to each state in the DFA a unique label, using the label
\(\lambda\) for the start state.

If the DFA transitions from the state with label \(S\) to the state with
label \(T\) when it consumes character \(s\), add the following rule to
the rule sequence:

\begin{tikzpicture}
[vertex/.style={circle, draw, fill=black, inner sep=0pt, minimum width=4pt}]

\node[vertex] (s) [label=left:$S$] {};
\node[vertex] (t) [label=left:$a$] at (0,-1) {};
\node[vertex] (c) [label=right:$b_s$] at (1,-1) {};

\node[vertex] (s2) [label=left:$c$] at (4,0) {};
\node[vertex] (t2) [label=left:$T$] at (4,-1) {};
\node[vertex] (c2) [label=right:$c$] at (5,-1) {};

\draw[thick] (s2) -- (t2);
\draw[thick] (s2) -- (c2);
\draw[thick] (c2) -- (t2);

\draw[-{Straight Barb[left]}] (2,-0.5) -- (3,-0.5);

\end{tikzpicture}

where \(c\) is a reserved label, not corresponding to any of the states
of the DFA.

Additionally, add the following rule to the rule sequence:

\begin{tikzpicture}
[vertex/.style={circle, draw, fill=black, inner sep=0pt, minimum width=4pt}]

\node[vertex] (c) [label=below:$c$] {};
\node[vertex] (d) [label=below:$c$] at (1,0) {};

\node[vertex] (c2) [label=below:$c$] at (4,0) {};
\node[vertex] (d2) [label=below:$c$] at (5,0) {};

\draw[thick] (c2) -- (d2);

\draw[-{Straight Barb[left]}] (2,0) -- (3,0);

\end{tikzpicture}

\subsection{Citations}\label{citations}

\end{document}
